\documentclass{llncs}

\title{Bold Hearts Team Description\\RoboCup 2013 Kid Size}
\author{Sander van Dijk \and Drew Noakes \and Daniel Polani}

\institute{Adaptive Systems Research Group\\School of Computer Science\\University of Hertfordshire, UK}

\begin{document}

\maketitle

%
%         - includes a statement committing to participate in the
%           RoboCup 2013 Humanoid League competition,
%         - includes a statement committing to making a person
%           with sufficient knowledge of the rules available as referee
%           during the competition,
%         - fully describes the scientific aspects of your humanoid robot
%           system and your research interests,
%         - includes a summary of previous relevant achievements in
%           research and development as well as publications,
%         - mentions prior performance in RoboCup competitions
%         - points out enhancements of the robots' hardware or software
%           compared to the previous year
%         - states explicitly whether software from other teams is used
%           and if yes, which parts are used and what the team's own
%           contributions are
%

\begin{abstract}
\end{abstract}

\section{Introduction}
\label{sec:introduction}

Statement committing to participate. Introduce the Bold Hearts;
summarize the history and achievements; describe members as varying
from undergrad to doctorate students to industry and staff; give
introduction on main goals and research interests of the team.
Statement committing supplying a referee.

\section{Simulation \& Hardware, Joint Opportunities}
\label{sec:simul-amp-hardw}

Make an introductory case of how a cross-league effort can benefit
all: we bring high level strategy to hardware league, and can feed
back realism into simulation league (mention team members have been
active in several levels of 3DSSL organization?). Note to our
knowledge we are the first team coming from simulation league (some
are in SPL though).

\section{Research \& Development}
\label{sec:research}

\subsection{Shared Framework}
\label{sec:framework}

Mention we use libbats: mature but actively developed open source base
framework in 3DSSL; used by several successful teams; Bold Hearts
currently main maintainers; in the process of porting to hardware, to
supply common medium level control interface (skill set framework);
challenges (and describe how we solve it): vision (well on way to have
real time high resolution solution), noise + ... observations (libbats
natively allows for different localization solutions), different +
uncertain model (again, libbats allows for this).

\subsection{Bipedal Locomotion}
\label{sec:bipedal-locomotion}

Describe inverse kinematic gait control; describe optimization in
simulation; describe energy limiting optimization + publication
\cite{lattarulo_application_2011}.

\section{Achievements}
\label{sec:achievements}

List awards

\bibliographystyle{plain}
\bibliography{tdp}

\end{document}

