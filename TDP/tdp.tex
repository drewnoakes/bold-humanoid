\documentclass{llncs}

\title{Bold Hearts Team Description\\RoboCup 2013 Kid Size}
\author{Sander van Dijk \and Drew Noakes \and Daniel Polani}

\institute{Adaptive Systems Research Group\\School of Computer Science\\University of Hertfordshire, UK}

\begin{document}

\maketitle

%
%         - includes a statement committing to participate in the
%           RoboCup 2013 Humanoid League competition,
%         - includes a statement committing to making a person
%           with sufficient knowledge of the rules available as referee
%           during the competition,
%         - fully describes the scientific aspects of your humanoid robot
%           system and your research interests,
%         - includes a summary of previous relevant achievements in
%           research and development as well as publications,
%         - mentions prior performance in RoboCup competitions
%         - points out enhancements of the robots' hardware or software
%           compared to the previous year
%         - states explicitly whether software from other teams is used
%           and if yes, which parts are used and what the team's own
%           contributions are
%

\begin{abstract}
  In this paper we describe the RoboCup Humanoid Kid Size section of
  team Bold Hearts, the RoboCup team of the University of
  Hertfordshire, in Hatfield UK. As a team that is well experienced in
  the 3D Soccer Simulation, we focus on combining our efforts in both
  teams, by sharing frameworks and experience between both. 
\end{abstract}

\section{Introduction}
\label{sec:introduction}

Team Bold Hearts, representing the University of Hertfordshire in
Hatfield, UK, is committed to participate in the 2013 RoboCup Humanoid
League, Kid Size championship in Eindhoven, The Netherlands. The team
was founded in 2002, competing in both the 2D and 3D RoboCup Simulation
Leagues. After a short hiatus, a full restart of the team was initiated.
As listed below, this resulted in several successful years with a number
of rewards.

The team consists of members with a wide range of experience, from
first year undergraduate students to master students, PhD
students, professional developers from industry and university
staff.

Team Bold Hearts has a long history within the 3D simulation league, competing
in thirteen events since 2009, placing in the top four in all but one.
Two of our team members have served on the technical committee, and one as
the organising chair in both a regional and world cup competition.
All team members participate voluntarily on an extra-curricular basis.

The University of Hatfield procured seven DARwIn-OP robots at the end
of 2012. Our intention is to compete at a high level in both the 2013
German Open in Magdebourg, and the 2013 World Cup in Eindhoven.

The main goals of our team are twofold: firstly, to provide a
valuable source of experience, contacts, and enjoyment.
Secondly, to use the robotic football scenario as an important
test bed for several of our ongoing research projects.

In the rest of this paper we go into these aspects more
deeply. Finally, team Bold Hearts commits to supplying at least one
team member with enough knowledge of the Humanoid Kid Size rules to
perform as referee during the competitions.

\section{Simulation \& Hardware, Joint Opportunities}
\label{sec:simul-amp-hardw}

To the best of our knowledge team Bold Hearts is currently the only team
working on both Soccer Simulation and Kid Size challenges.

In previous years, SimSpark, the 3D simulator, has been used to model
a replica of the Nao robot from the Standard Platform League. 2013
promises to be the first year in which heterogeneous robot models
will be used during competitions.

As we work to bring the capabilities of our 3D team to physical robots,
we are undertaking the expansion of our codebase to cover multiple diverse platforms.
We have found the rapid prototyping of ideas and automation of learning algorithms and
optimisation that the simulator afford very valuable and are working
to utilise such techniques within the Kid Size league.

We also have considerable experience of developing the
3D simulator itself, including RoboCup Federation-funded
project work. We are learning more about the challenges physical
robots pose, and will strive to use this knowledge to improve the
capability and fidelity of the simulation where it makes sense to do so.

Each league has much to offer, and we are pleased to have taken a
position that spans such challenges and disciplines.

%Make an introductory case of how a cross-league effort can benefit
%all: we bring high level strategy to hardware league, and can feed
%back realism into simulation league (mention team members have been
%active in several levels of 3DSSL organization?). Note to our
%knowledge we are the first team coming from simulation league (some
%are in SPL though).

\section{Research \& Development}
\label{sec:research}

In this section we will go into more detail on the research and
development that our team is interested in and actively working
on. In the first part we will describe the framework we have built for
the simulation league and are currently porting to the physical
robots, and discuss the benefits and challenges in doing so. Secondly,
we review our work on bipedal locomotion, an important topic for both
leagues.

\subsection{Shared Framework}
\label{sec:framework}

As a base for our team we use the {\tt libbats} framework
\footnote{http://launchpad.net/libbats}. This library was originally
developed by 3D simulation team Little Green BATS, from the University
of Groningen, The Netherlands, in 2006. It has been released as
open-source, and currently team Bold Hearts are the maintainers.
Several other teams have used it to base their work on.

Besides a low level interface to the simulation environment, the
library provides a range of modules, including localisation through
Kalman or particle filters, forward and inverse kinematics, graphical
monitoring and debugging tools, a generic skill-set interface,
formation and coordination frameworks, and an extensive XML
configuration module. We are currently in the process of porting this
over to the DARwIn-OP platform.

In doing so, we encounter several challenges that arise from
connecting the different leagues. For instance, the vision model in
the simulated environment is more abstract. As this is such an
important source of information, our first work consists implementing
image processing methods needed to feed the framework with similar
data. This involves dealing with higher levels of noise, more
ambiguity, and false positives and negatives. This makes the Kalman
localization method used in simulation unfeasible, but luckily {\tt
  libbats} natively supports using different localization
implementations.

Another major difference are the different physical aspects of the
simulated Nao and physical DARwIn-OP robots. Again, {\tt libbats} is
set up to easily configure different body models. Also, the shared
framework, and the open architecture of the RoboCup SSL 3D simulation
server, will allow us to quickly set up a simulated test-bed for the
Kid Size league to prototype, test, and optimize in.

\subsection{Bipedal Locomotion}
\label{sec:bipedal-locomotion}

Much of the success of team Bold Hearts in the RoboCup SLL 3D
competitions was due to high speed of locomotion. Throughout the years we have gained much experience in setting up 


For the 2012 competitions in Mexico, we have developed a completely new inverse-kinematics based gait controller


 Describe inverse
kinematic gait control; describe optimization in simulation; describe
energy limiting optimization + publication
\cite{lattarulo_application_2011}.

\section{Achievements}
\label{sec:achievements}

2012
3rd RoboCup World Championship 2012, Mexico City, Mexico
2nd Dutch Open 2012, Eindhoven, The Netherlands
4th Iran Open 2012, Qazvin, Iran
2011
Top 8 RoboCup World Championship 2011, Istanbul, Turkey
1st Iran Open 2011, Tehran, Iran
3rd German Open 2011, Magdeburg, Germany
2010
2nd AUTCUP 2010, Tehran, Iran
2nd RC4EW, Eisteddfod of Wales 2010, Ebbw Vale, UK
4th RoboCup World Championship 2010, Singapore
1st German Open 2010, Magdeburg, Germany
3rd Iran Open 2010, Tehran, Iran
2009
2nd RoboCup World Championship 2009, Graz, Austria
1st German Open 2009, Hannover, Germany

\bibliographystyle{plain}
\bibliography{tdp}

\end{document}
