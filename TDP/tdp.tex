\documentclass{llncs}

\title{Bold Hearts Team Description\\RoboCup 2013 Kid Size}
\author{Sander van Dijk \and Drew Noakes \and Daniel Polani}

\institute{Adaptive Systems Research Group\\School of Computer Science\\University of Hertfordshire, UK}

\begin{document}

\maketitle

%
%         - includes a statement committing to participate in the
%           RoboCup 2013 Humanoid League competition,
%         - includes a statement committing to making a person
%           with sufficient knowledge of the rules available as referee
%           during the competition,
%         - fully describes the scientific aspects of your humanoid robot
%           system and your research interests,
%         - includes a summary of previous relevant achievements in
%           research and development as well as publications,
%         - mentions prior performance in RoboCup competitions
%         - points out enhancements of the robots' hardware or software
%           compared to the previous year
%         - states explicitly whether software from other teams is used
%           and if yes, which parts are used and what the team's own
%           contributions are
%

\begin{abstract}
  In this paper we describe the RoboCup Humanoid Kid Size section of
  team Bold Hearts, the RoboCup team of the University of
  Hertfordshire, in Hatfield UK. As a team that is well experienced in
  the 3D Soccer Simulation, we focus on combining our efforts in both
  teams, by sharing frameworks and experience between both. 
\end{abstract}

\section{Introduction}
\label{sec:introduction}

Team Bold Hearts, representing the University of Hertfordshire, in
Hatfield, UK, is committed to participate in the 2013 RoboCup Humanoid
League, Kid Size championship in Eindhoven, The Netherlands. The team
was founded in 2002, working on the RoboCup Simulation League, both
the 2D as the 3D competitions. After a short hiatus, in a full restart
of the team was initiated. As listed below, this resulted in several
successful years with a number of rewards.

The team consists of members with a wide range of experience, from
first year undergraduate students, to master students and PhD
students, to professional developers from industry and university
staff.

The main goals of our team are twofold: firstly, all students
participate on an extra-curricular base, and as such it offers a great
source of experience, contacts, and enjoyment to their
degrees. Secondly, we aim to us the robotic football scenario as an
important test bed for several of our ongoing research projects.

In the rest of this paper we go into these aspects more
deeply. Finally, team Bold Hearts commits to supplying at least one
team member with enough knowledge of the Humanoid Kid Size rules to
perform as referee during the competitions.

\section{Simulation \& Hardware, Joint Opportunities}
\label{sec:simul-amp-hardw}

For as far as we know, team Bold Hearts is currently the only team
working both on the Soccer Simulation and Kid Size challenges. 

Make an introductory case of how a cross-league effort can benefit
all: we bring high level strategy to hardware league, and can feed
back realism into simulation league (mention team members have been
active in several levels of 3DSSL organization?). Note to our
knowledge we are the first team coming from simulation league (some
are in SPL though).

\section{Research \& Development}
\label{sec:research}

\subsection{Shared Framework}
\label{sec:framework}

Mention we use libbats: mature but actively developed open source base
framework in 3DSSL; used by several successful teams; Bold Hearts
currently main maintainers; in the process of porting to hardware, to
supply common medium level control interface (skill set framework);
challenges (and describe how we solve it): vision (well on way to have
real time high resolution solution), noise + ... observations (libbats
natively allows for different localization solutions), different +
uncertain model (again, libbats allows for this).

\subsection{Bipedal Locomotion}
\label{sec:bipedal-locomotion}

Describe inverse kinematic gait control; describe optimization in
simulation; describe energy limiting optimization + publication
\cite{lattarulo_application_2011}.

\section{Achievements}
\label{sec:achievements}

List awards

\bibliographystyle{plain}
\bibliography{tdp}

\end{document}

